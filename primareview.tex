\documentclass[11pt]{beamer}
\usepackage[utf8]{inputenc}
\usepackage[T1]{fontenc}
\usepackage{lmodern}
\usepackage[italian]{babel}
\usetheme{Boadilla}
\begin{document}
	\author{Ciraolo, Porta, Rizzi}
	\title{Prima review\\INA System}
	%\subtitle{}
	%\logo{}
	%\institute{}
	%\date{}
	%\subject{}
	%\setbeamercovered{transparent}
	%\setbeamertemplate{navigation symbols}{}
	\begin{frame}[plain]
		\maketitle
	\end{frame}
	
	\begin{frame}
		\frametitle{Project goals and NO goals}
		
		\begin{columns}
			\begin{column}{.45\linewidth}
				\begin{itemize}
					\item Semplificare la risoluzione di debiti e crediti in gruppi di fiducia
					\item Pianificazione e votazione di eventi o parte degli stessi
					\item Creazione di gruppi modulari con preferenze su creditori e debitori
					\item Integrazione bot con Telegram e simili
				\end{itemize}
			\end{column}
			\begin{column}{.45\linewidth}
				\begin{itemize}
					\item Non è un "portafogli" e non tiene valore economico
					\item Non gestisce i pagamenti fra gli utenti
					\item Non implementa un sistema per trovare partecipanti
				\end{itemize}
			\end{column}
		\end{columns}
	\end{frame}

	\begin{frame}
		\frametitle{Motivations}
		
		\begin{itemize}
			\item Comodità nella gestione dei gruppi
			\item Facilità di saldo di crediti e debiti data la risoluzione ottimizzata
			\item Mantenimento di \textit{prove di acquisto}
			\item Automatizzazione della risoluzione
		\end{itemize}
	\end{frame}

	\begin{frame}
		\frametitle{Similar applications}
		
		\begin{itemize}
			\item \href{https://play.google.com/store/apps/details?id=com.Splitwise.SplitwiseMobile&hl=en_US&gl=US}{Splitwise}
			\item \href{https://play.google.com/store/apps/details?id=com.tribab.tricount.android}{Tricount}
			\item \href{https://play.google.com/store/apps/details?id=org.marbot.travel.money.free}{Travel Money}
			\item \href{https://play.google.com/store/apps/details?id=com.jwang123.splitbills}{Split Bills}
		\end{itemize}
	\end{frame}

	\begin{frame}
		\frametitle{Initial Project plan summary}
	\end{frame}

	\begin{frame}
		\frametitle{Status as of November 15, 2021}
	\end{frame}
	
	\begin{frame}
		\frametitle{User stories}
	\end{frame}
	
	\begin{frame}
		\frametitle{Spikes to be done}
	\end{frame}
	
	\begin{frame}
		\frametitle{User Interaction Draft}
	\end{frame}
	
	\begin{frame}
		\frametitle{CRC Cards – First draft}
	\end{frame}
	
	\begin{frame}
		\frametitle{Mockups}
	\end{frame}

	\begin{frame}
		\frametitle{Domande?}
		TODO SLIDE DA SISTEMARE
	\end{frame}
\end{document}