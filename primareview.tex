\documentclass[11pt]{beamer}
\usepackage[utf8]{inputenc}
\usepackage[T1]{fontenc}
\usepackage{lmodern}
\usepackage[italian]{babel}
\usetheme{Boadilla}
\date{15 Novembre 2021}
\begin{document}
	\author{Ciraolo, Porta, Rizzi}
	\title[IiNA System]{Prima review\\IiNA System}
	%\subtitle{}
	%\logo{}
	%\institute{}
	%\date{}
	%\subject{}
	%\setbeamercovered{transparent}
	%\setbeamertemplate{navigation symbols}{}
	\begin{frame}[plain]
		\maketitle
	\end{frame}
	
	\begin{frame}
		\frametitle{Project goals and NO goals}
		
		\begin{columns}
			\begin{column}{.45\linewidth}
				\begin{center}
					\textbf{Goals}
				\end{center}
				\begin{itemize}
					\item Semplificare la risoluzione di debiti e crediti in gruppi di fiducia
					\item Pianificazione e votazione di eventi o parte degli stessi
					\item Creazione di gruppi modulari con preferenze su creditori e debitori
					\item Integrazione bot con Telegram e altre app di messaggistica
				\end{itemize}
			\end{column}
			\begin{column}{.45\linewidth}
				\begin{center}
					\textbf{No Goals}
				\end{center}
				\begin{itemize}
					\item Non è un "portafogli" e non tiene valore economico
					\item Non gestisce i pagamenti fra gli utenti%\footnotemark[1]
					\item Non implementa un sistema per trovare partecipanti
				\end{itemize}
			\end{column}
		\end{columns}
%		\footnotetext[1]{Potenziale aggiunta futura}
	\end{frame}

	\begin{frame}
		\frametitle{Motivations}
		
		\begin{itemize}
			\item Comodità nella gestione dei gruppi
			\item Facilità di saldo di crediti e debiti data la risoluzione ottimizzata
			\item Mantenimento di \textit{prove di acquisto}
			\item Automatizzazione della risoluzione
			\item Planning del viaggio o degli eventi
			\item Decisioni condivise e democratiche
			\item Organizzazione di grandi gruppi, eventualmente modulari
			\item[] \includegraphics[width=.1\linewidth]{gnu.png}
		\end{itemize}
	\end{frame}

	\begin{frame}
		\frametitle{Similar applications}
		
		\begin{itemize}
			\item \href{https://play.google.com/store/apps/details?id=com.Splitwise.SplitwiseMobile&hl=en_US&gl=US}{Splitwise}
			\item \href{https://play.google.com/store/apps/details?id=com.tribab.tricount.android}{Tricount}
			\item \href{https://play.google.com/store/apps/details?id=org.marbot.travel.money.free}{Travel Money}
			\item \href{https://play.google.com/store/apps/details?id=com.jwang123.splitbills}{Split Bills}
		\end{itemize}
	\end{frame}

	\begin{frame}
		\frametitle{Initial Project plan summary}
		% TODO: telegram bot development
		\begin{columns}
			\begin{column}{.45\linewidth}
				\begin{itemize}
					\item Novembre
					\begin{itemize}
						\item Spikes
						\item Analisi componenti critici 
						\item Modellazione delle funzionalità principali
						\item Mockups dettagliati
						\item Sviluppo iniziale del core
					\end{itemize}
					\item 15/12/21
					\begin{itemize}
						\item Gestione dei gruppi a livello singolo
						\item Elezioni admin
						\item Primi stadi web app e android app
					\end{itemize}
				\end{itemize}
			\end{column}
			\begin{column}{.45\linewidth}
				\begin{itemize}
					\item Fine Dicembre
					\begin{itemize}
						\item Gruppi multilivello
						\item Debiti individuali
						\item Board
						\item Secondo stadio delle applicazioni
					\end{itemize}
					\item Metà gennaio
					\begin{itemize}
						\item Pools e planning
						\item Refactoring and polishing
						\item Completamento prima release applicazioni
					\end{itemize}
				\end{itemize}
			\end{column}
		\end{columns}
		
%		\begin{itemize}
%			\item Novembre
%			\begin{itemize}
%				\item Spikes
%				\item Analisi componenti critici 
%				\item Modellazione delle funzionalità principali
%				\item Mockups dettagliati
%				\item Sviluppo iniziale del core
%			\end{itemize}
%			\item 15/12/21
%			\begin{itemize}
%				\item Gestione dei gruppi a livello singolo
%				\item Elezioni admin
%				\item Primi stadi web app e android app
%			\end{itemize}
%			\item Fine Dicembre
%			\begin{itemize}
%				\item Gruppi multilivello
%				\item Debiti individuali
%				\item Board
%				\item Secondo stadio delle applicazioni
%			\end{itemize}
%			\item Metà gennaio
%			\begin{itemize}
%				\item Pools e planning
%				\item Refactoring and polishing
%				\item Completamento prima release applicazioni
%			\end{itemize}
%		\end{itemize}
	\end{frame}

	\begin{frame}
		\frametitle{Status as of November 15, 2021}
		
		\begin{itemize}
			\item User stories
			\item Alcuni spikes
			\item Mockups
			\item Bozza di interazioni
			\item CRC cards
		\end{itemize}
	\end{frame}

	\section{User stories}
	
	\begin{frame}
		\begin{itemize}
			\item \textbf{Creazione del gruppo}
			
			\item[] Stiamo partendo in vacanza noi 10 del gruppo e abbiamo deciso di gestire le spese per ridurre al minimo il giro di piccoli pagamenti fra noi.
			
			Creiamo un gruppo in cui aggiungiamo tutti i partecipanti alla gita e poniamo un budget per la vacanza come indicatore.
			
			\item \textbf{Aggiunta costo sostenuto}
			
			\item[] Mentre stiamo andando verso il Monte Bianco iniziamo a pagare il primo pedaggio, siamo divisi in tre auto e il costo è fisso per tutte.
			
			Subito dopo aver pagato aggiungiamo i tre costi sostenuti al sistema con una descrizione degli stessi. Viene aggiunta la data e posso allegare la foto della ricevuta.
		\end{itemize}
	\end{frame}

	\begin{frame}
		\begin{itemize}
			\item \textbf{Aggiunta credito uno a uno}
			
			\item[] Mario ha pagato i 18 euro di pedaggio ma gli mancavano le monete e gliele ho fornite io, per cui ho aggiunto al sistema questo credito verso Mario; alla risoluzione ne verrà tenuto in conto.
			
			Serve solo che lui confermi questa transazione per aggiungerla allo storico.
			
			\item \textbf{Risoluzione dei debiti}
			
			\item[] La vacanza è finita e vanno saldati i conti, chiudiamo il gruppo ed il sistema risolve i debiti in modo da avere la lista finale di debiti ottimizzata. Ovvero con ciascuno solo debitore o creditore e un numero di transazioni ridotte al minimo.
		\end{itemize}
	\end{frame}

	\begin{frame}
		\begin{itemize}
			\item \textbf{Votazioni e organizzazione}
			
			\item[] Nel prepare la scampagnata dobbiamo decidere cosa portare in tavola. Siamo in tanti e l'organizzazione risulta complicata.
			
			Creo il gruppo come amministratore e apro una bacheca; in essa ciascuno può inserire se si impegna a portare qualcosa ed in seguito aggiungerà il costo sostenuto.
			
			Prima però dobbiamo decidere tutti insieme cosa mangiare per cui creo dei sondaggi sulle varie portate e ciascuno può esprimere la preferenza.
			
			\item \textbf{Amministrazione}
			
			\item[] Deciso il cibo mi rendo conto che non ho spazio a sufficienza in casa per tenere anche le bevande per cui chiedo a qualcun altro di amministare la gestione delle bibite.
			
			Possiamo tutti votare per un nuovo admin, che si candidato, e chi prende più voti diventa il nostro \textit{sommelier}.
		\end{itemize}
	\end{frame}

	\begin{frame}
		\begin{itemize}
			\item \textbf{Fusione di gruppi}
			
			\item[] Lucia si laurea la prossima settimana e noi del gruppo dell'Università abbiamo deciso di fare un regalo insieme; nel frattempo però il suo gruppo di amici da casa ci ha contattato proponendo di fare un regalo più grande insieme.
			
			A questo punto abbiamo deciso di unire il nostro gruppo regalo già fatto al loro e provare ad organizzare il regalo, sia nella decisione che nella raccolta dei fondi, tutti insieme.
			
			Nell'unire i due gruppi vengono considerati i partecipanti in modo che il costo sia diviso equamente ad eccezione di chi vuol contribuire di più per i fatti loro.
			
			Gli amministratori dei due gruppi, se assenti vengono eletti, diventano i responsabili per il proprio gruppo e fra loro avviene, se possibile, il saldo economico fra i due gruppi.
			
			Le votazioni sono come sempre e ciascuno ha un voto unitario.
		\end{itemize}
	\end{frame}
	
	\begin{frame}
		\frametitle{Spikes to be done}
		
		\begin{itemize}
			\item Spring
			\item Android
			\item Docker
			\item Kubernetes
			\item OAuth
			\item Telegram bot
		\end{itemize}
	\end{frame}
	
	\begin{frame}
		\frametitle{User Interaction Draft}
	\end{frame}
	
	\begin{frame}
		\frametitle{CRC Cards – First draft}
	\end{frame}
	
	\begin{frame}
		\frametitle{Mockups}
	\end{frame}

	\begin{frame}
		\frametitle{Domande?}
		TODO SLIDE DA SISTEMARE
	\end{frame}
\end{document}