\documentclass[10pt,a4paper]{article}
\usepackage[utf8]{inputenc}
\usepackage[T1]{fontenc}
\usepackage[italian]{babel}
\usepackage{amsmath}
\usepackage{amsfonts}
\usepackage{amssymb}
\usepackage{graphicx}
\author{Ciraolo, Porta, Rizzi}
\title{TAASS\\User stories}
\date{AA 21/22}
\begin{document}
	\maketitle
	
%	\section{Basics}
%	
%	\begin{itemize}
%		\item Gruppi possono avere o non avere amministratori
%		\item Utenti di gruppi \textit{plain} hanno stessi privilegi
%		\item Amministratori possono \textit{forzare} alcuni elementi
%		\item Gruppi valutabili come \textit{persone} pesate
%		\item Gestione democratica, votazioni e preferenze
%	\end{itemize}
	
%	\section{Plain group}
	
	\section{Creazione}
	
	Stiamo partendo in vacanza noi 10 del gruppo e abbiamo deciso di gestire le spese per ridurre al minimo il giro di piccoli pagamenti fra noi.
	
	Creiamo un gruppo in cui aggiungiamo tutti i partecipanti alla gita e poniamo un budget per la vacanza come indicatore.
	
	\section{Aggiunta costo sostenuto}
	
	Mentre stiamo andando verso il Monte Bianco iniziamo a pagare il primo pedaggio, siamo divisi in tre auto e il costo è fisso per tutte.
	
	Subito dopo aver pagato aggiungiamo i tre costi sostenuti al sistema con una descrizione degli stessi. Viene aggiunta la data e posso allegare la foto della ricevuta.
	
	\section{Aggiunta credito uno a uno}
	
	Mario ha pagato i 18 euro di pedaggio ma gli mancavano le monete e gliele ho fornite io, per cui ho aggiunto al sistema questo credito verso Mario; alla risoluzione ne verrà tenuto in conto.
	
	Serve solo che lui confermi questa transazione per aggiungerla allo storico.
	
	\section{Risoluzione dei debiti}
	
	La vacanza è finita e vanno saldati i conti, chiudiamo il gruppo ed il sistema risolve i debiti in modo da avere la lista finale di debiti ottimizzata. Ovvero con ciascuno solo debitore o creditore e un numero di transazioni ridotte al minimo.
	
%	\section{Planning}
	
	\section{Votazioni e organizzazione}
	
	Nel prepare la scampagnata dobbiamo decidere cosa portare in tavola. Siamo in tanti e l'organizzazione risulta complicata.
	
	Creo il gruppo come amministratore e apro una bacheca; in essa ciascuno può inserire se si impegna a portare qualcosa ed in seguito aggiungerà il costo sostenuto.
	
	Prima però dobbiamo decidere tutti insieme cosa mangiare per cui creo dei sondaggi sulle varie portate e ciascuno può esprimere la preferenza.
	
	\section{Amministrazione}
	
	Deciso il cibo mi rendo conto che non ho spazio a sufficienza in casa per tenere anche le bevande per cui chiedo a qualcun altro di amministare la gestione delle bibite.
	
	Possiamo tutti votare per un nuovo admin, che si candidato, e chi prende più voti diventa il nostro \textit{somalier}.
	
%	\section{Modules}
	
	\section{Fusione di gruppi}
	
	Lucia si laurea la prossima settimana e noi del gruppo dell'Università abbiamo deciso di fare un regalo insieme; nel frattempo però il suo gruppo di amici da casa ci ha contattato proponendo di fare un regalo più grande insieme.
	
	A questo punto abbiamo deciso di unire il nostro gruppo regalo già fatto al loro e provare ad organizzare il regalo, sia nella decisione che nella raccolta dei fondi, tutti insieme.
	
	Nell'unire i due gruppi vengono considerati i partecipanti in modo che il costo sia diviso equamente ad eccezzione di chi vuol contribuire di più per i fatti loro.
	
	Gli amministratori dei due gruppi, se assenti vengono eletti, diventano i responsabili per il proprio gruppo e fra loro avviene, se possibile, il saldo economico fra i due gruppi.
	
	Le votazioni sono come sempre e ciascuno ha un voto unitario.
	
	
\end{document}